\documentclass[microtype,a4paper,article,10pt,oneside,final]{memoir}

\usepackage[utf8x]{inputenc}
\usepackage[hyperref,svgnames]{xcolor}
\usepackage[xetex]{hyperref}
\usepackage{xspace}
\usepackage[a4paper,scale=0.8]{geometry}

%opening
\title{BI-MAP User Guide}
\author{Alexey Stukalov}
\date{October 2011}

\hypersetup{pdftitle={BI-MAP User Guide},
            colorlinks=true, linkcolor=DarkSlateGray, citecolor=DarkBlue, urlcolor=NavyBlue}

\counterwithout{section}{chapter}

\newcommand{\bimap}{\texttt{BI-MAP}\xspace}

\begin{document}

\maketitle

\setcounter{tocdepth}{3}
\tableofcontents
\vspace*{1cm}

\section{Software Requirements}

To run \bimap the following packages need to be installed:
\begin{itemize}
  \item \href{http://www.boost.org}{Boost} 1.43+ for extended set of C++ templates
  \item \href{http://www.gnu.org/s/gsl/}{GSL} 1.10+ for mathematical and statistical routines
  \item \href{http://r-project.org}{R} 2.13+ for results analysis and visualization (optional)
  \item \href{http://dirk.eddelbuettel.com/code/rcpp.html}{Rcpp} 0.9+ for loading \bimap results into R session (optional)
  \item \href{http://www.open-mpi.org/}{MPI Implementation} 1.5+ for parallel version of \bimap sampler (optional)
\end{itemize}

\bimap is distributed both as LGPL-licensed source code and precompiled binaries package.
The users of binary package could skip the compilation section.

\section{Compilation}

\bimap requires \href{http://www.cmake.org}{CMake} 2.6+ build system for compiling from the source code.

Compilation requires the ``development'' versions of the packages listed at software requirements section to be installed.
For unit tests compilation and executing \href{http://code.google.com/p/googletest/}{GoogleTest} 1.5+ is required.

To initialize the build environment, run
\begin{verbatim}
> cmake
\end{verbatim}
from the command prompt of root \bimap folder.
It should automatically locate all required packages.
Refer to the CMake manual for more options.
Upon successful execution \texttt{cmake} generates standard UNIX Makefiles.
To start the compilation, execute
\begin{verbatim}
> make
\end{verbatim}

\section{Installation}

Simply unpack the contents of \bimap binary package.
The following binaries are provided:
\begin{itemize}
  \item \texttt{libBIMAP.so} -- dynamic library with core \bimap functions
  \item \texttt{BIMAP-sampler} -- single-process \bimap executable
  \item \texttt{BIMAP-sampler-mpi} -- multi-process \bimap sampler that uses MPI
  \item \texttt{libRBIMAP.so} -- R $\leftrightarrow$ \bimap interface dynamic library
\end{itemize}

\section{Running}

\subsection{Input Data}

For modules inference \bimap requires the description of 
AP-MS experiments, experimental data
and some properties of the identified proteins.
This should be provided in 3 plain-text files (tab-separated tables),
see table~\ref{table:params:input}.

\vspace{0.2cm}

\begin{table}[h]
\caption{Parameters for Input Files specification}
\begin{tabular}{rp{0.3\textwidth}p{0.4\textwidth}}
\bimap option & Description & Example \\
\hline
\verb|--exp_design_file|
& Experimental design table:
  Bait Accession Code,
  Biological Sample Identifier,
  MS run Identifier,
  MS run normalization constant ($m(e)$)
&
\vspace{-0.5cm}
\begin{verbatim}
Bait_AC  Sample_ID  MSrun_ID     mult
Prot1    Prot1_WT   Prot1_WT-1   1.0
Prot1    Prot1_WT   Prot1_WT-2   0.9
Prot1    Prot1_Mut  Prot1_Mut-1  0.5
Prot2    Prot2_WT   Prot2_WT-1   1.0
...
\end{verbatim}
\\
\verb|--measurements_file|
& MS spectral counts table:

MS run identifier,
protein Accession Code, spectral counts
&
\vspace{-0.5cm}
\begin{verbatim}
MSrun_ID    Prey_AC  SC
Prot1_WT-1  Prot1    500
Prot1_WT-1  Prot3    10
Prot2_WT-1  Prot1    5
Prot2_WT-1  Prot2    600
...
\end{verbatim}
\\
\verb|--proteins_file|
& Proteins table:

Accession Code, AA Sequence Length
&
\vspace{-0.5cm}
\begin{verbatim}
AC     SeqLength
Prot1  1000
Prot2  1200
...
\end{verbatim}
\\
\verb|--input_file|
&
\emph{Optional}.
XML file with combined information\footnotemark[1].
\end{tabular}
\label{table:params:input}
\end{table}

\footnotetext[1]{Alternatively, required information could be prepared within R session and serialized into by \texttt{OPASaveData()} function.
See \texttt{samples/tip49/analysis.R} script for further details of preparing AP-MS data and running \bimap within R session.}

\subsection{Method and Model Parameters}

\bimap provides access to most of its parameters.
Parameters could be specified at command-line
or be supplied in .INI file via \verb|--config_file| option.

There are the following groups of parameters
(correspond to sections in .INI file or serve as dot-separated prefix
to the parameter name when specified at command line):
\begin{enumerate}
  \item \texttt{prior}. Parameters for prior model (Pitman-Yor distribution for proteins and experiments etc).
  \item \texttt{hyperprior}. Hyperprior model parameters (normal-scaled inverse-gamma distribution for protein module abundance).
  \item \texttt{signal}. Parameters of spectral counts distribution (\texttt{signal\_counts\_shape} -- Lagrangian Poisson distribution shape parameter,
\texttt{seq\_length\_factor} -- $\beta_L$ parameter in (8)).
  \item \texttt{precomputed}. \texttt{object}/\texttt{probe\_freq\_threshold} parameters control
the maximum frequency of proteins and experiments to be used for the calculation of
preliminary protein--protein and experiment--experiment distances.
  \item \texttt{cascade}. Parameters of the equi-energy sampler: number of processes (\emph{turbines}),
parameters of temperature ladder, size of intermediate steps cache etc (see \texttt{--help}).
  \item \texttt{sampler}. Parameters for Gibbs sampling steps and the rates of the steps.
\end{enumerate}

To get the description of all available \bimap options, run
\begin{verbatim}
> BIMAP-sampler --help
\end{verbatim}

\begin{table}[h]
\caption{Sampler parameters affecting \bimap execution times and the quality of the biclustering models distribution}

\begin{tabular}{rp{0.5\textwidth}r}
Command-line Parameter & Description & Default value \\
\hline

\verb|--cascade.burnin_iterations|
& number of \emph{burn-in} iterations to estimate initial parameters of the model.
Longer burn-in period might be required for larger datasets to ensure that near-optimal values have been found.
& $2 \cdot 10^5$ \\
\verb|--samples| & number of samples to collect & 1000 \\
\verb|--storage_period| & period (\# of sampler iterations) before new sample is collected.
Longer period might be required for larger datasets to ensure statistical independence of
subsequently collected samples.
& 1000 \\
\verb|--priors_storage_period| & period (\# of cluster samples collected)
before new sample of prior parameters is collected.
Longer period might be required for larger datasets to ensure statistical independence.
& 5
\end{tabular}
\end{table}

\begin{table}[h]
\caption{Model parameters affecting matrix blocks granularity}

\begin{tabular}{p{0.2\textwidth}p{0.5\textwidth}r}
Parameter & Description & Default value \\
\hline

{\scriptsize\texttt{--prior.obj/probe\_ clustering\_ concentration/discount}}
& $(\theta_P, \alpha_P)$, $(\theta_E, \alpha_E)$ --
concentration and discount parameters for 
Pitman-Yor prior distribution for rows and columns.
Larger concentration values increase the number of clusters & (0.3, 0.3)
\\
{\scriptsize\texttt{--signal.signal\_counts\_shape}}
& $\lambda$, shape parameter of Lagrangian Poisson spectral counts distribution.
Larger values reduce sensitivity to spectral counts and result in larger matrix blocks.
& 0.1
\\
{\scriptsize\texttt{--precomputed.object/probe \_freq\_threshold}} &
Threshold to exclude frequent proteins/experiments
from the calculation of preliminary protein-protein and
experiment-experiment distances.
The optimal threshold depends on the cleanliness of the dataset.
& 0.7
\end{tabular}
\end{table}

The recommended values for the other parameters are provided in \texttt{samples/tip49/config.ini}.

\section{Results Analysis and Visualization}

The output of \texttt{BIMAP-sampler} (or \texttt{BIMAP-sampler-mpi})
is a collection of chessboard biclustering models generated by MCMC sampler.
The name of the output file is specified by \verb|--output_file| option.

\bimap comes with a set of scripts to analyse the sampling results.
\begin{itemize}
  \item \texttt{BIMAP.R} -- \bimap running, \bimap results reading,
individual biclustering extraction
  \item \texttt{BIMAP\_plot.R} -- matrix plot of chessboard biclustering models
  \item \texttt{BIMAP\_graphml.R} -- export biclustering model to GraphML
  \item \texttt{ms\_data\_exttools.R} -- reading NestedCluster results and conversion into chessboard biclustering models
\end{itemize}

These scripts require \texttt{libRBIMAP.so}.
Before loading these scripts into R session,
\texttt{RBIMAP.libpath} variable should be set to point to the location of \texttt{libRBIMAP.so}.

Some of the key functions, provided by these scripts:
\begin{itemize}
\item \texttt{BIMAP.walk.load()} loads the output of \bimap.
\item \texttt{BIMAP.extract\_clustering()} extracts the biclustering with specified ID,
\item \texttt{plot.bimap()} visualizes extracted \bimap model as a block matrix
\item \texttt{bimap.graphML()} converts extracted \bimap model into network representation and exports to GraphML format
(could be read by graph editors like \href{http://yed.yworks.com}{yEd}).
\end{itemize}

The sample R script that reads and analyses TIP49 dataset is \texttt{samples/tip49/analysis.R}.

\end{document}
